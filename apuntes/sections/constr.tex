\section{Constructores de tipos}

Dados los tipos básicos podemos combinarlos para construir nuevos tipos en otras
estructuras como listas, árboles o tuplas. Podemos pensar en los constructores de
tipos como funciones sobre tipos: Toman uno o más tipos y devuelven un nuevo tipo.

\begin{extra}{Sinonimos de tipo}
\texttt{String} es un sinónimo de tipo de \texttt{[Char]}:
son distintos nombres para el mismo tipo. La notación usual para
cadenas es equivalente a la de listas:

\espacio

\texttt{"hi" == ['h','i']}

\espacio

Los sinónimos de tipo se definen con la palabra clave \texttt{type}:

\espacio

\texttt{type String = [Char]}
\end{extra}

Veamos algunos ejemplos:

\subsection{Listas}
Las listas almacenan colecciones ordenadas de valores de un tipo homogéneo.
Su tamaño \textbf{no es fijo}, es decir, listas de distintos tamaños pertenecen
al mismo tipo.

Una lista se escribe entre corchetes, separando sus elementos por comas.
La lista vacía se escribe \texttt{[]}:

\begin{lstlisting}
[1,2,3]   :: [Int]
[]        :: [a]
['a','b'] :: [Char]
\end{lstlisting}

También podemos utilizar el constructor \texttt{:},
que antepone un elemento al principio de una lista:
\begin{lstlisting}
[1,2,3] == 1:2:3:[]
\end{lstlisting}

\begin{extra}{Rangos}
Haskell proporciona una sintaxis de rangos para tipos enumerables, que permite
definir listas fácilmente. Prueba estos ejemplos:

\begin{itemize}
  \item \texttt{[1..10]}
  \item \texttt{['a'..'m']}
  \item \texttt{[1,4..20]}
  \item \texttt{[10,8..1]}
\end{itemize}

Su implementación se basa en la clase de tipos \texttt{Enum}.
\end{extra}

Algunas funciones relevantes sobre listas (y otras estructuras \footnote{\texttt{null}, \texttt{elem} y \texttt{length} funcionan sobre otro tipo de estructuras de datos como conjuntos, árboles o vectores en versiones recientes de GHC. \texttt{<>} funciona sobre cualquier semigrupo.}) son:

\begin{itemize}
  \item \texttt{<> :: [a] -> [a] -> [a]}: concatena dos listas.
  \item \texttt{head :: [a] -> a}: devuelve la cabeza (primer elemento).
  \item \texttt{tail :: [a] -> [a]}: devuelve la cola.
  \item \texttt{null :: [a] -> Bool}: indica si una lista está vacía.
  \item \texttt{elem :: Eq a => a -> [a] -> Bool}: indica si un elemento está en la lista.
  \item \texttt{(!!) :: [a] -> Int -> a}: elemento en la posición dada.
  \item \texttt{length :: [a] -> Int}:
  \footnote{La mayoría de funciones sobre listas están restringidas sobre \texttt{Int}
  por razones de retrocompatibilidad.} indica la longitud.
\end{itemize}

Otras funciones generalizan una función binaria sobre una lista u otras estructuras. Es el caso de
\texttt{sum}, \texttt{product}, \texttt{maximum} o \texttt{minimum}.

\subsection{Tuplas}
Las tuplas son un tipo que almacena varios datos en una sola estructura.
Podemos pensar en ellas como el producto cartesiano de dos tipos.

Podemos guardar por ejemplo una película y su año:

\begin{lstlisting}
(1960, "Psicosis") :: (Int, String)
\end{lstlisting}

Las tuplas tienen un tamaño \textit{fijo}. Cada n-tupla es un tipo diferente.
Pueden contener elementos de distintos tipos.

Es usual utilizar los pares (2-tuplas), como en la función \texttt{zip}.
Esta función toma dos listas y devuelve una lista de pares con los elementos de
cada lista:

\begin{lstlisting}
  zip :: [a] -> [b] -> [(a,b)]
\end{lstlisting}

No existen tuplas de un elemento, aunque sí existe la tupla vacía, escrita \texttt{()}
y llamada \textit{unit}, que funciona como \texttt{None} en Python o \texttt{void} en C++.
