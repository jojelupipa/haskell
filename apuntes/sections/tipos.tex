\section {El sistema de tipos}

En Haskell están definidos los tipos básicos comunes a otros lenguajes,
y pueden usarse directamente.
Podemos comprobar el tipo de cualquier expresión con \texttt{:t}. \footnote{
Algunos de estos ejemplos están simplificados; se describen en la sección de
clases de tipos.} Por ejemplo:

\begin{lstlisting}
True   :: Bool
'a'    :: Char
"a"    :: String
10^320 :: Integer
2      :: Int
3.3    :: Double
\end{lstlisting}

Los tipos de Haskell son \textbf{estáticos} y \textbf{fuertes}.
El compilador hace comprobaciones de tipo e infiere los tipos cuando no se
declaran explícitamente. La mayoría de errores de un código escrito en Haskell
serán de tipos; en otro caso, sólo pueden ser errores de sintaxis o lógicos.

\begin{otro}{Tipos fuertes y estáticos}
  En un sistema de tipos estáticos el tipo de todos los elementos es conocido en tiempo de compilación.
  Es el caso de C++, pero no el de Python o Javascript.

\espacio

Un sistema de tipos fuertes impide la conversión implícita entre tipos. En C++
los tipos no son fuertes: podemos convertir un entero a un booleano
de forma implícita para comprobar si es nulo.
\end{otro}


\subsection{Polimorfismo}

Algunos datos pueden tener más de un tipo. En estos casos, Haskell asume
el tipo más general e infiere el tipo concreto al usarlo.
Podemos indicar el tipo de los objetos que usemos para
facilitar la depuración.

\begin{lstlisting}
fst :: (a,b) -> a
id  :: a -> a
\end{lstlisting}

Sobre enteros la identidad tendría tipo \texttt{id :: Int -> Int}, y sobre
booleanos \texttt{id :: Bool -> Bool}. Haskell infiere el tipo más general: puede
tomar un tipo cualquiera (\texttt{a}) y devuelve un elemento de ese tipo. La letra
\texttt{a} es una \textbf{variable de tipo}.

\begin{extra}{Variables de tipo}
Las variables de tipo deben empezar con una letra \textbf{minúscula}.
Usualmente se utiliza una sola letra, pero puede
utilizarse cualquier identificador que empiece por minúscula.

\espacio

En contraposición, los tipos concretos empiezan por una letra
\textbf{mayúscula} (\texttt{Bool} y no \texttt{bool}). Si los escribimos en
minúscula el compilador pensará que son una variable de tipo.
\end{extra}

\subsection{Clases de tipos}

Algunos tipos tienen propiedades comunes. Para aprovecharlas se definen
\textbf{clases de tipos}, que agrupan tipos con la misma interfaz.
La mayoría de los tipos son instancias de \texttt{Eq},
porque disponen de una función \texttt{(==)} \footnote{Y una función \texttt{(/=)}
para comprobar la desigualdad.}.
Las instancias de la clase \texttt{Num} pueden sumarse y multiplicarse,
las de \texttt{Show} pueden convertirse a texto (\texttt{String}),
y las de \texttt{Integral} permiten calcular restos modulares sobre ellas.

\newpage

Las clases de tipos se indican utilizando \texttt{=>}:

\begin{lstlisting}
3    :: Num a => a
show :: Show a => a -> String
gcd  :: Integral a => a -> a -> a
(==) :: Eq a => a -> a -> Bool
\end{lstlisting}

Un tipo puede pertenecer a varias clases: \texttt{Integer} pertenece a todas las
clases anteriores.

\begin{extra}{Tipos numéricos}
Los números en Haskell son polimórficos: \texttt{3} puede ser interpretado como
\texttt{Double} o \texttt{Int}. Esto permite mantener un sistema de tipos fuerte
sin preocuparse por convertirlos explícitamente.

\espacio

Los tipos más comúnmente utilizados son:

\begin{itemize}
  \item \texttt{Int}: Enteros de 32 o 64 bits.
  \item \texttt{Integer}: Precisión ilimitada.
  \item \texttt{Double}: Punto flotante.
\end{itemize}

\espacio

Las clases \texttt{Num}, \texttt{Integral} y \texttt{Fractional} unifican ciertas
propiedades de los tipos numéricos. Puedes leer más
\enlace{https://wiki.haskell.org/Converting_numbers}{aquí}.
\end{extra}

\subsection{Primeras definiciones}

Todo objeto en Haskell es una función: una constante es simplemente una función
sin argumentos:

\begin{lstlisting}
constante = 28
\end{lstlisting}

Definimos las funciones con una serie de igualdades que cubran todas las formas
posibles de los argumentos de entrada. El compilador intentará encajar los
argumentos en alguna de esas formas en orden y usará la \textbf{primera}
definición que encaje. Veamos un ejemplo:

\begin{lstlisting}
muestra :: Bool -> String
muestra True  = "True"
muestra False = "False"
\end{lstlisting}

\begin{extra}{Librerías}
Cargamos librerías con \texttt{import <nombre>}, tanto en un archivo como en
\texttt{ghci}. Si no están ya instaladas podemos hacerlo utilizando
\texttt{cabal}.

\espacio

Para buscar una función concreta existe \enlace{http://www.haskell.org/hoogle/}{Hoogle},
que permite buscar una función por nombre o por tipo.
\end{extra}

Podemos poner opcionalmente antes de la definición el tipo. Esto se llama
\textbf{declaración de tipo}. En la mayoría de las ocasiones Haskell es capaz
de deducir el tipo de una función:

\begin{lstlisting}
duplica x = x + x
\end{lstlisting}

Cuando usemos variables estas tienen que empezar por \textbf{minúscula} o ser
\texttt{\_} \footnote{Este es el único nombre de variable que puede utilizarse
más de una vez}, que utilizamos para indicar que el argumento no se va a utilizar:

\begin{lstlisting}
esNulo :: Int -> Bool
esNulo 0 = True
esNulo _ = False

esPar x = esNulo (mod x 2)
\end{lstlisting}


\subsection {Funciones, tipos y currificación}
Las funciones en un lenguaje funcional son tratadas como objetos
de primera clase. Pueden pasarse como argumentos o ser devueltas como resultado.
Por defecto una función en Haskell es currificada.

La currificación es la identificación entre las funciones:
\begin{align*}
\begin{matrix}
f: A \times B \to C    & \quad &   f: A \to (B \to C) \\
f(a,b) = c       & \quad &     f(a)(b) =  c
\end{matrix}
\end{align*}

En lugar de tener una función que toma dos argumentos y devuelve
un resultado, tenemos una función que toma un argumento y devuelve otra
función que toma otro argumento para devolver el resultado.
Por ejemplo, la suma queda:
\begin{align*}
\begin{matrix}
(+) :: & \R \times \R & \rightarrow & \R    & \quad & (+) :: & \R & \rightarrow & (\R \rightarrow  \R) \\
       & x , y      & \mapsto     & x+y  & \quad &        & x & \mapsto     & (y  \mapsto x+y)
\end{matrix}
\end{align*}

De esta forma una función binaria toma un argumento y devuelve una función
unaria. Por inducción se definen funciones de más argumentos, toman un sólo argumento
y devuelven
\footnote{Consideramos $\rightarrow$ asociativo por la derecha, de tal forma que
\texttt{a -> b -> c} es equivalente a \texttt{a -> (b -> c)}}
una función con un argumento menos:

\begin{lstlisting}
(&&) :: Bool -> Bool -> Bool
(+) :: Num a => a -> a -> a
const :: a -> b -> a
(==) :: Eq a => a -> a -> Bool
\end{lstlisting}

Podemos indicar sólo una parte de los argumentos de una función. Esto se conoce
como \textbf{aplicación parcial}. Aquí definimos la función
\texttt{xor} y definimos la negación sobre ella, usando primero la definición
normal y luego usando aplicación parcial:

\begin{lstlisting}
xor :: Bool -> Bool -> Bool
xor x y = (not x && y) || (x && not y)

neg' x = xor True x -- Sin aplicación parcial
neg = xor True      -- Con aplicación parcial
\end{lstlisting}

\begin{extra}{Combinando funciones}
Existen 3 operadores muy utilizados para combinar funciones:

\espacio

\begin{itemize}
  \item \texttt{(f . g) x = f (g x)}
  \item \texttt{flip f x y = f y x}
  \item \texttt{f \$ x = f x}
\end{itemize}

\espacio

\texttt{.} es la composición de funciones y \texttt{flip} cambia el orden de
los argumentos.

\texttt{\$} es el operador con menor prioridad, lo que permite ahorrar paréntesis:

\espacio

\texttt{succ (9 + 10)} $\rightarrow$ \texttt{succ \$ 9 + 10}

\end{extra}
De la misma forma podemos llamar \texttt{(+)} con sólo un argumento, y nos
devolverá la función que suma ese número:

\begin{lstlisting}
suma5 = (5+)
mult2 = (2+)
\end{lstlisting}

Podemos componer funciones con la función composición, \texttt{(.)} y aprovechar
la aplicación parcial para omitir argumentos.

\begin{lstlisting}
f = suma5 . mult2 -- f x = 2*x + 5
\end{lstlisting}

\newpage
