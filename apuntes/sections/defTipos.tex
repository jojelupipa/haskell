\subsection{Definición de tipos}
Para definir un tipo utilizamos \texttt{data}. \footnote{Existen otras formas
de definir tipos. Puedes leer más
\enlace{https://en.wikibooks.org/wiki/Yet_Another_Haskell_Tutorial/Type_advanced\#Newtypes}{aquí.}}
Por ejemplo, el tipo \texttt{Bool} se define:

\begin{lstlisting}
  data Bool = False | True
\end{lstlisting}

Después de \texttt{data} indicamos el nombre del tipo y después de \texttt{=} indicamos los
\textit{constructores de datos}, separados por \texttt{|}. Tanto el nombre de tipo como
los constructores de datos deben empezar con mayúscula. \footnote{Excepto si son un
símbolo como en los tipos numéricos o en las listas.} Podemos pensar en un tipo como
\texttt{Int} como si estuviera definido:

\begin{fullwidth}
\begin{lstlisting}
data Int = -9223372036854775808 | ... | -1 | 0 | 1 | ... | 9223372036854775807
\end{lstlisting}
\end{fullwidth}

Podemos definir tipos con campos que dependan de otros tipos, como por ejemplo
puntos en el espacio:

\begin{lstlisting}
data Point = Point Int Int Int
\end{lstlisting}

Como vemos un tipo puede llamarse igual que uno de sus constructores y
para indicar un campo de un cierto tipo ponemos el tipo en el lugar del campo.
Los tipos pueden definirse recursivamente, como en el caso de los naturales
según los axiomas de Peano:

\begin{lstlisting}
data Nat = Zero | Succ Nat
\end{lstlisting}

Un natural puede ser el cero o un sucesor de un natural.

También podemos definir \textbf{constructores de tipos}. Podemos verlos como
funciones sobre tipos: toman como argumento uno o más tipos y definen nuevos
tipos. Es el caso del tipo lista o de las tuplas
\footnote{Las definiciones de listas y tuplas son primitivas para poder usar
esta notación. Podríamos definir tipos equivalentes utilizando sintaxis legal.}:

\begin{lstlisting}
data [a] = [] | a : [a]
data (a,b) = (a,b)
\end{lstlisting}

Después del nombre del constructor de tipos damos un nombre a las variables que
utiliza (\texttt{a}). Podemos ver entonces que  \texttt{[a]} puede
ser una lista vacía (\texttt{[]}) o un par de elementos, el primero de tipo
\texttt{a} y el segundo una lista de tipo \texttt{a}.

Los constructores de datos se utilizan para definir funciones en lo que se conoce
como \textit{reconocimiento de patrones}. Cuando definimos una función podemos
emplear una variable para un argumento o bien un constructor de datos.
Por ejemplo\footnote{Podemos usar el mismo nombre para las variables de tipo y
los nombres de los elementos del constructor porque sus ámbitos son distintos.}:

\begin{lstlisting}
fst :: (a,b) -> a
fst (a,b) = a
\end{lstlisting}

Como vimos antes en el caso de las listas es habitual utilizar el constructor
\texttt{:}
\footnote{Debemos poner \texttt{head (x:xs)} y no \texttt{head x:xs} ya que si
no nuestra función estaría tomando 3 argumentos: \texttt{x}, \texttt{:} y \texttt{xs}
}:

\begin{lstlisting}
head (x:xs) = x
second (x:y:xs) = y
\end{lstlisting}

Otro ejemplo es \texttt{(<>)}\footnote{En realidad, esta definición debería estar dentro de una instancia de \texttt{Semigroup}}:
\begin{lstlisting}
(<>) :: [a] -> [a] -> [a]
[]     <> ys = ys
(x:xs) <> ys = x : (xs <> ys)
\end{lstlisting}

Podemos utilizar el reconocimiento de patrones sobre cualquier tipo que soporte
la igualdad utilizando cualquier constructor del tipo. Debemos asegurarnos de que
cubrimos todos los constructores.
