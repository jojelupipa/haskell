\section{Funciones}

\begin{frame}[fragile]{Reconocimiento de patrones}
  Para definir una función sobre un tipo, definimos su comportamiento para cada
  constructor de datos del tipo:

  \espacio

  \begin{lstlisting}
neg :: Bool -> Bool
neg False = True
neg True  = False
  \end{lstlisting}

  \espacio
  
  Podemos utilizar variables para sustituir cualquier argumento del constructor:

  \espacio

  \begin{lstlisting}
suma (Point a b) (Point c d) = Point (a+c) (b+d)
opuesto (Point a b) = Point (-a) (-b)
resta a b = suma a (opuesto b)
  \end{lstlisting}
\end{frame}

\begin{frame}[fragile]{Recursividad}
  El tipo lista y las definiciones recursivas son la base de los programas de Haskell.
  Por ejemplo, para calcular la \textbf{longitud de una lista}, definimos la función
  \texttt{len} para sus dos constructores:

  \espacio

  \begin{lstlisting}
len []     = 0
len (_:xs) = 1 + len xs
  \end{lstlisting}

    \espacio

  El primer constructor nos proporciona el caso base y el segundo la ecuación recursiva.
\end{frame}
