\section{Haskell}

\begin{frame}[fragile]{Instalando Haskell Platform}
  El paquete \texttt{haskell-platform} contiene el compilador, depurador, gestor de
  librerías y otras utilidades para programar en Haskell.
  En otras distribuciones puede instalarse directamente \texttt{ghc}
  (Glasgow Haskell Compiler):
  \espacio

  \begin{lstlisting}[language=bash]
sudo apt-get install haskell-platform
  \end{lstlisting}
\end{frame}

\begin{frame}[fragile]{Sin efectos secundarios}
    En los lenguajes de programación funcional, una función toma un argumento y
    devuelve una salida. No altera el mundo alrededor ni cambia el valor de los argumentos.

    \espacio

    En lenguajes imperativos, cuando llamamos a una función puede cambiar nuestras
     variables o escribir por pantalla. Eso hace que el orden de llamada de las
    funciones importe.

    \espacio

    % programmers.stackexchange.com/questions/40297/
  \begin{lstlisting}[language=C++]
int n = 0;
int next_n() { return n++; }
next_n(); // n = 1
  \end{lstlisting}
\end{frame}

\begin{frame}{El intérprete: GHCi}
  GHC es un compilador de Haskell con GHCi como intérprete asociado.
  El intérprete permite los siguientes comandos:
  \begin{itemize}
    \item \texttt{:q} \qquad  Quitar
    \item \texttt{:l} \qquad  Cargar módulo
    \item \texttt{:r} \qquad  Recargar módulos
    \item \texttt{:t} \qquad  Consultar tipos
  \end{itemize}

  Una vez cargado el intérprete podemos utilizarlo para probar el lenguaje.
  Haskell permite operaciones aritméticas básicas, y operaciones con
  cadenas, listas o booleanos.
\end{frame}

\begin{frame}[fragile]{El intérprete: GHCi}
  Podemos probar el uso de un puñado de funciones simples. Las funciones
  se escriben dejando sus argumentos a su lado y separados por espacios. ¡Estamos usando
  \textbf{notación polaca}!

  \begin{lstlisting}
ghci> 3 + 4
7
ghci> (+) 2 9
11
ghci> succ 27
28
ghci> max 23 34
34
  \end{lstlisting}
\end{frame}
