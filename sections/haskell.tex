\section{Haskell}

\begin{frame}[fragile]{Instalando Haskell Platform}
  \texttt{haskell-platform} contiene el compilador, depurador y otras utilidades.
  También podemos instalar \texttt{ghc}:
  \espacio
  \begin{lstlisting}[language=bash]
sudo apt-get install haskell-platform
  \end{lstlisting}
  \espacio
  Ambos traen un gestor de librerías: \texttt{cabal}.
  \note{
    \begin{itemize}
      \item ghc: (Glorious) Glasgow Haskell Compiler.
      \item Linter: \texttt{hlint}.
    \end{itemize}}
\end{frame}

\begin{frame}[fragile]{Sin efectos secundarios}

    Las funciones en los lenguajes funcionales no tienen \textit{efectos secundarios}.
    No alteran el mundo alrededor ni cambian el valor de los argumentos.
    \espacio
  \begin{lstlisting}[language=C++]
int n = 0;
int next() { return n++; }
next(); // n = 1
  \end{lstlisting}

  \note{
  \begin{itemize}
    \item Ejemplo de \url{http://programmers.stackexchange.com/questions/40297}.
    \item Una función puede cambiar variables o escribir por pantalla.
    Eso hace que el orden de llamada importe.
  \end{itemize}
  }
\end{frame}

\begin{frame}[fragile]{Las funciones como objetos}
 Las funciones son objetos de \textit{primera clase}. Pueden ser devueltos
 por funciones y pueden pasarse como argumentos.

 \begin{lstlisting}
 duplica lista = map (\ x -> 2*x) lista
 \end{lstlisting}

\end{frame}

\begin{frame}{El intérprete: GHCi}
  GHC incluye GHCi como intérprete. Permite los siguientes comandos:
  \espacio
  \begin{itemize}
    \item \texttt{:q} \qquad  Quitar
    \item \texttt{:l} \qquad  Cargar módulo
    \item \texttt{:r} \qquad  Recargar módulos
    \item \texttt{:t} \qquad  Consultar tipos
  \end{itemize}

\note{
Haskell permite operaciones aritméticas básicas, y operaciones con
cadenas, listas o booleanos.
\begin{itemize}
  \item \texttt{:set +t} para mostrar el tipo por defecto.
  \item \texttt{:set +m} para permitir entrada de varias líneas.
\end{itemize}

}
\end{frame}

\begin{frame}[fragile]{El intérprete: GHCi}
 Las funciones se llaman escribiendo su nombre, un espacio y sus parámetros, separados por espacios:
\espacio
  \begin{lstlisting}
ghci> 3 + 4
7
ghci> (+) 2 9
11
ghci> succ 27
28
ghci> max 23 34
34
  \end{lstlisting}

\note{
\begin{itemize}
  \item Estamos usando \textbf{notación polaca}.
  \item Para escribir una función infija de forma prefija se pone entre paréntesis.
  \item Para escribir una función prefija de forma infija se pone entre acentos graves.
\end{itemize}
}
\end{frame}
