\section{Ejemplos}

\begin{frame}[fragile]{Quicksort}
  Implementación del algoritmo Quicksort
  \begin{lstlisting}[language=haskell]
qsort []     = []
qsort (x:xs) = qsort [y | y<-xs, y<=x]
            ++ [x]
            ++ qsort [y | y<-xs, y>x]
  \end{lstlisting}
\end{frame}

\begin{frame}[fragile]{Árboles binarios}
  Los definimos como vacíos o un nodo con dos árboles:
  \begin{lstlisting}
 data Tree a = Empty
             | Node a (Tree a) (Tree a)
  \end{lstlisting}
  \ejemplo{Node 4 (Node 3 Empty Empty) Empty}

  \espacio

  Esto nos deja definir funciones sobre ellos fácilmente:
  \begin{lstlisting}
 preorder :: Tree a -> [a]
 preorder Empty = []
 preorder (Node x a b) =
      x : preorder a ++ preorder b
  \end{lstlisting}
\end{frame}
