\section{Comprensión de listas y rangos}
\subsection{Rangos}
Los rangos son azúcar sintáctico de listas para crear listas que contengan sucesiones de elementos
de la clase \texttt{Enum}:

\begin{lstlisting}
[1..10] -- [1,2,3,4,5,6,7,8,9,10]
['a'..'m'] -- "abcdefghijklm"
[1,4..20] -- [1,4,7,10,13,16,19]
[10,8..1] -- [10,8,6,4,2]
\end{lstlisting}

Esta clase contiene los objetos que pueden ser enumerados. Puede usarse con \texttt{Double}, pero
no es recomendable por los errores de coma flotante.

\subsection{Comprensión de listas}
La comprensión de listas es una forma de escribir listas parecida a la notación para
expresar conjuntos por extensión. Su sintaxis es:
\begin{lstlisting}
-- [<expresion> | <variable> <- <lista>, <predicado>]
[x*2 | x <- [1..10]] -- [2,4,6,8,10,12,14,16,18,20]
[x*2 | x <- [1..10], x*2 >= 12] -- [12,14,16,18,20]
\end{lstlisting}

Cada elemento de la lista es sustituido en la expresión, que puede depender de la variable, y, si
cumple los predicados se añadirá a la lista. Si se utilizan varios predicados éstos se separan por
comas. También podemos utilizar varias listas, cada una ligada a una variable. En este caso las
variables se sustituirán por cada combinación posible de los elementos:
\begin{lstlisting}
[ (x,y) | x <- [1,2,3], y <- [8,3], x < y]
-- [(1,8),(1,3),(2,8),(2,3),(3,8)]
\end{lstlisting}

\section{Listas infinitas y evaluación perezosa}
La sintaxis de rangos nos permite escribir listas infinitas:

\begin{lstlisting}
[1..] -- Enteros positivos
fibs = 0 : 1 : zipWith (+) fibs (tail fibs) -- Secuencia de Fibonacci
repeat :: a -> [a] -- Crea listas infinitas con todos los elementos iguales
repeat n = n : repeat n -- repeat n = [n,n..]
\end{lstlisting}

Si intentamos mostrar estas listas en el intérprete no acabará. Sin embargo, podemos utilizarlas
junto con otras funciones sin problema:

\begin{lstlisting}
  zip [1..] ["azul","verde","rojo"] -- [(1,"azul"),(2,"verde"),(3,"rojo")]
\end{lstlisting}

Esto sucede porque Haskell es un lenguaje de evaluación no estricta o perezoso, por lo que sólo evaluará
los elementos que necesite. Así, en \texttt{||}, el segundo elemento sólo se evaluará si el
primero es \texttt{False}, ya que si es \texttt{True} el valor de la expresión ya estará fijado.
