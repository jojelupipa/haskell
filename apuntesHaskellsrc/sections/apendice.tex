\appendix

\section{Apéndice}

En este apéndice se incluyen otros temas de interés sobre la sintaxis o uso de
Haskell.

\subsection{Comprensión de listas}

La comprensión de listas es una forma de escribir listas parecida a la notación para
expresar conjuntos por extensión. Su sintaxis es:
\begin{lstlisting}
-- [<expresion> | <variable> <- <lista>, <predicado>]
[x*2 | x <- [1..10]] -- [2,4,6,8,10,12,14,16,18,20]
[x*2 | x <- [1..10], x*2 >= 12] -- [12,14,16,18,20]
\end{lstlisting}

Cada elemento de la lista es sustituido en la expresión, que puede depender de la variable, y, si
cumple los predicados se añadirá a la lista. Si se utilizan varios predicados éstos se separan por
comas. También podemos utilizar varias listas, cada una ligada a una variable. En este caso las
variables se sustituirán por cada combinación posible de los elementos:
\begin{lstlisting}
[ (x,y) | x <- [1,2,3], y <- [8,3], x < y]
-- [(1,8),(1,3),(2,8),(2,3),(3,8)]
\end{lstlisting}

\section{Listas infinitas y evaluación perezosa}
La sintaxis de rangos nos permite escribir listas infinitas:

\begin{lstlisting}
[1..] -- Enteros positivos
fibs = 0 : 1 : zipWith (+) fibs (tail fibs) -- Secuencia de Fibonacci
repeat :: a -> [a] -- Crea listas infinitas con todos los elementos iguales
repeat n = n : repeat n -- repeat n = [n,n..]
\end{lstlisting}

Si intentamos mostrar estas listas en el intérprete no acabará. Sin embargo, podemos utilizarlas
junto con otras funciones sin problema:

\begin{lstlisting}
  zip [1..] ["azul","verde","rojo"] -- [(1,"azul"),(2,"verde"),(3,"rojo")]
\end{lstlisting}

Esto sucede porque Haskell es un lenguaje de evaluación no estricta o perezoso, por lo que sólo evaluará
los elementos que necesite. Así, en \texttt{||}, el segundo elemento sólo se evaluará si el
primero es \texttt{False}, ya que si es \texttt{True} el valor de la expresión ya estará fijado.


\subsection{Prueba del compilador de Haskell}

Escribimos el clásico programa \textit{"Hello World"} para Haskell,
que servirá para probar el compilador.
En \texttt{test.hs} escribimos el programa:

\begin{lstlisting}
main = putStrLn "Hello World!"
\end{lstlisting}

Compilamos usando \texttt{ghc -o hello test.hs}
\footnote{También podríamos usar \texttt{runghc}, que compila y ejecuta el programa
en un solo paso}.
Acabamos de definir la función \texttt{main} como la función \texttt{putStrLn}
aplicada sobre la cadena \texttt{``Hello World!''}.

A diferencia de lo que haríamos en un lenguaje imperativo no le decimos que
para ejecutar \texttt{main} haya que ejecutar una función \texttt{putStr},
sino que \textbf{definimos} \texttt{main} como la función.

Además del compilador, \textit{Haskell Platform} incluye \texttt{runhaskell}, un
intérprete no interactivo que podríamos llamar con \texttt{runhaskell test.hs}
para este ejemplo.

\subsection{Ejemplos prácticos y más}

En la carpeta \texttt{examples} hay algunos ejemplos prácticos de programas
completos. Si quieres seguir aprendiendo, recomendamos los siguientes recursos:

\begin{itemize}
  \item \enlace{http://stackoverflow.com/questions/1012573}{Getting started with Haskell}
  \item \enlace{http://tryhaskell.org/}{Try Haskell}
  \item \enlace{http://aprendehaskell.es/}{¡Aprende Haskell por el bien de todos!}
  \item \enlace{https://en.wikibooks.org/wiki/Haskell}{Wikibooks - Haskell}
  \item \enlace{https://wiki.haskell.org/Learning_Haskell}{Learning Haskell}
  \item \enlace{http://book.realworldhaskell.org/}{Real World Haskell}
\end{itemize}
